A primeira parte da introdução deve conter um resumo do assunto a ser abordado na revisão bibliográfica daquilo que for o mais importante para o trabalho (neste caso, o controle alternativo de braços robóticos e redes neurais) 

\section{Motivação}%
Falar sobre o acoplamento de braços robóticos em cadeiras para melhor inserção de pessoas tetraplégicas no dia a dia da sociedade apta e a dificuldade de controlá-lo sem movimentos, especialmente finos, dos membros superiores. 
\\Falar também sobre a melhora na capacidade computacional e de armazenamento de dados permitindo que redes profundas com conjuntos expressivos de dados de treinamento para tornar o viável o uso de redes neurais para solução de problemas. 
\\Juntar as duas partes da motivação e falar sobre como a Mecatrônica voltada para tecnologias assistivas e engenharia biomédica é uma boa área. 

\section{Problema}
Falar sobre o problema central que a pesquisa desenvolvida busca resolver.
\\Exemplo: "Uma maneira alternativa de controlar apêndices mecânicas ou robóticas tem sido procurada, e necessária, ha algum tempo. Este trabalho busca apresentar uma solução que não exige a fixação de equipamento diretamente no usuário, utilizando uma câmera e técnicas de processamento de imagens."

\section{Objetivos}
Falar o objetivo principal: "Propor um método de controle de braços robóticos utilizando imagens e redes neurais."
\\Listar também objetivos intermediários/específicos:

\begin{itemize}
    \item revisão bibliográfica da literatura sobre imagens, sinais, redes neurais, eletrônica;
    \item aquisição e pre-processamento dos dados;
    \item algoritmo de treinamento e estrutura de rede;
    \item comunicação do computador de processamento das imagens com o elemento mecânico/robótico;
    \item relatar os resultados.
\end{itemize}%

\section{Descrição dos capítulos}
/auto-explicativo, feito ao final/